\documentclass[11pt]{article}
%\usepackage{times}
%\input latexdef.tex

\textheight=9.25in %\textwidth=7in
\textwidth=6.50in  % wide margin for paper review and room for labels
\topmargin=-24pt
\oddsidemargin= -.25in
\evensidemargin=0pt
\overfullrule=0pt
%\thispagestyle{empty}


\newcommand{\bfs}{{\mathbf{s}}}
\newcommand{\bfw}{{\mathbf{w}}}
\newcommand{\bfy}{{\mathbf{y}}}
\newcommand{\bfone}{{\mathbf{1}}}

\newcommand{\Expected}{\mathrm{I\!E}}


\begin{document}

\setcounter{page}{1}
\thispagestyle{empty}

%%%%%%%%%%%%%%%%%%%%%%%%%%%%%%%%%%%%%%%%%%%%%%%%%%%%%%%%%%%

%%\renewcommand{\baselinestretch}{2.0}

\begin{center}
{\bf Stat 641 Bayesian Statistics}

Homework 9, due Monday, November 23, 2020.
\end{center}
\noindent
\begin{enumerate}
\item Sampling from prior distributions. If you specify in your JAGS model statement a model that
doesn't have a likelihood (that is, it refers to no observations), you can obtain samples from the
probability distribution of whatever else is stated in your model. (In such situations, it doesn't 
matter whether we call it a ``prior distribution'' or simply a ``distribution''; we're effectively
using software that's designed for Bayesian models to do something that wasn't necessarily intended.)
We might wish to do this -- fit a model with no likelihood -- to make sure we understand what
the distributions are that JAGS is using: is the 2nd parameter the variance or the precision?
Is the mean $\alpha/\beta$ or is it $\alpha \beta$? Etc.

\begin{enumerate}
\item Use JAGS to simulate 10,000 samples from what-it-calls a Gamma(3,5) distribution using this model
statement:
\begin{verbatim}
model{
  theta ~ dgamma(3,5)
}
\end{verbatim}
Use the JAGS output (\verb+summary(my.coda.samples)+) to answer these
questions:
Are the mean and variance of this distribution $\alpha/\beta$ and $\alpha/\beta^2$?
 Or are they $\alpha\beta$ and $\alpha/\beta^2$?

\item Use JAGS to simulate 10,000 samples from what-it-calls an InverseGamma(3,5) distribution using this model
statement:
\begin{verbatim}
model{
 foo ~ dgamma(3,5)
 theta <- 1/foo
}
\end{verbatim}
Use the JAGS output to answer these
questions: Are the mean and variance of this distribution 
$\beta/(\alpha-1)$
and $\beta^2/[(\alpha-1)^2(\alpha-2)]$?
 Or are they
$1/(\beta(\alpha-1))$
and  \\$1/[ \beta^2(\alpha-1)^2(\alpha-2)]$?

\item Use JAGS code to simulate 10,000 values from the distribution specified by the following model statement:
\begin{verbatim}
model{
 k ~ dcat( probs[] )
}
\end{verbatim}

Here's some R code for fitting this model. Be sure to run it one line at a time so
you can figure out what it's doing. (The Latex file is posted along with this
assignment, so you can copy and paste the R code.)

\begin{verbatim}
    library(rjags)
    my.data <- list( probs = 0.1*rep(1,10) )
    my.inits1 <- list( k = 1 )
    my.inits <- list( my.inits1 )
    my.fname <- "dcat-model.txt"
    my.jags.model <- jags.model(
      file = my.fname, data = my.data, inits = my.inits,
      n.chains = 1, n.adapt = 1000, quiet = FALSE )
    my.variables <- c("k")
    my.coda.samples <- coda.samples(
      my.jags.model,my.variables,100000,thin=1)
    summary(my.coda.samples)
    k.samples <- my.coda.samples[[1]][,"k"]
    unique( k.samples )
    hist(k.samples, prob=TRUE, main="")
    lines(density(k.samples))
    #plot(my.coda.samples)
\end{verbatim}


\begin{enumerate}
\item What is the distribution of $k$? What values does $k$ take, and what are the
associated probabilities?
\item Be sure to include the histogram so you can discuss it.
\item Use the statement below rather than the earlier code for my.data,
and refit the model.
What is the distribution of $k$? What values does $k$ take, and what are the
associated probabilities?

\begin{verbatim}
sum(1:8)
my.data <- list( probs = (1:8)/sum(1:8) )
\end{verbatim}
\end{enumerate}

\item Use JAGS code to simulate 10,000 values from the distribution specified by the following model statement:
\begin{verbatim}
model{
 foo ~ dcat( probs[] )
 k <- values[foo]
}
\end{verbatim}

Here's some R code for fitting this model. 

\begin{verbatim}
library(rjags)
probs <- c(1,2,3)/6
values <- c( 1, 42, 17 )
my.data <- list( probs = probs, values = values )
my.inits1 <- list( foo = 1 )
my.inits <- list( my.inits1 )
my.fname <- "dcat-model.txt"
my.jags.model <- jags.model(
  file = my.fname, data = my.data, inits = my.inits,
  n.chains = 1, n.adapt = 1000, quiet = FALSE )
my.variables <- c("k")
my.coda.samples <- coda.samples(
  my.jags.model,my.variables,100000,thin=1)
summary(my.coda.samples)
k.samples <- my.coda.samples[[1]][,"k"]
unique( k.samples )
hist(k.samples, prob=TRUE, main="")
lines(density(k.samples))
#plot(my.coda.samples)
\end{verbatim}


\begin{enumerate}
\item What is the distribution of $k$? What values does $k$ take, and what are the
associated probabilities?
\item Be sure to include the histogram so you can discuss it.
\end{enumerate}

\end{enumerate}

\item A changepoint model.  Changepoint models are models for data that
are measured over time, where the behavior of the data appears to change
at some point in time. For example, $Y_i$ = number of meals eaten at a restaurant in
week $i$, where the values were fairly steady pre-covid-19, and fairly steady post-covid-19,
but at a different level. The data set for this problem consists of 
counts of coal mining disasters in Great Britain, by year, from 1851 to 1962, which
is 112 years worth of data.
 (Here, ``disaster'' is defined as an accident resulting in the deaths 
of 10 or more miners.)
The data is posted on Blackboard in the file, \verb+coal_mining_disasters.txt+.

\begin{enumerate}
\item Plot the data; discuss briefly. Does it appear to you that the rate of
coal mining disasters changed appreciably during this time interval? If so, in 
approximately what year do you think the change began? Here's some R code.
Be sure to include the plot in your solutions.

\begin{verbatim}
my.data <- read.table("coal-mining-disasters.txt",header=TRUE)
names(my.data)
disasters <- my.data$disasters
year <- my.data$year
N <- length(disasters); N
plot(year,disasters, type='l')
\end{verbatim}

\item We consider the following changepoint model for these data.
$$Y_i \sim \left\{
\begin{array}{ll}
{\mbox{Poisson}}(\lambda_1), & i = 1,2,\ldots,k \\
{\mbox{Poisson}}(\lambda_2), & i = k+1,k+2,\ldots,n
\end{array}\right.$$
$$\lambda_i \stackrel{ind}{\sim} {\mbox{Exponential}}(1), \,\, i = 1,2$$

We initially assume $k=40$. What year is the coal mining disaster rate
assumed to have changed?

\item State the posterior distribution: $p( \lambda_1,\lambda_2 | \bfy ) \propto L( \lambda_1,
\lambda_2) \pi( \lambda_1,\lambda_2 ) \propto \cdots$

\item This will be our model statement.
\begin{verbatim}
model {
  for( i in 1:N ) {
    disasters[i] ~ dpois( lambda[ idx[i] ] )
    idx[i] <- 1+step( i-k-0.5 )
  }
  lambda[1] ~ dexp(1)
  lambda[2] ~ dexp(1)
}
\end{verbatim}

Explain carefully these two lines, which specify the likelihood:
\begin{verbatim}
    disasters[i] ~ dpois( lambda[ idx[i] ] )
    idx[i] <- 1+step( i-k-0.5 )
\end{verbatim}
For which values of $i$ will idx[i] be equal to 1? For which values of $i$ will
it be equal to 2? Are we sure that idx[i] will always be either 1 or 2, which it
needs to be in the line, \verb+disasters[i] ~ dpois( lambda[ idx[i] ] )+? Explain
briefly.

\pagebreak

\item Fit the model using JAGS. Here's some R code for fitting the model:

\begin{verbatim}
library(rjags)
my.data <- list( disasters = disasters, N = N, k = 40 )

my.inits1 <- list( lambda = c(40,40) )
my.inits2 <- list( lambda = c(20,20) )
my.inits3 <- list( lambda = c(100,10) )
my.inits <- list( my.inits1, my.inits2, my.inits3 )

my.model.1.fname <- "coal-mining-disasters-model-1.txt"
my.jags.model <- jags.model(
  file = my.model.1.fname, data = my.data, inits = my.inits,
  n.chains = 3, n.adapt = 1000, quiet = FALSE )

dic.samples(my.jags.model, n.iter=10000, thin=1, type="pD")
my.variables <- c("lambda" )

my.coda.samples <- coda.samples(my.jags.model,
                                my.variables,10000, thin=1)
lambda1.samples <- my.coda.samples[[1]][,"lambda[1]"]
lambda2.samples <- my.coda.samples[[1]][,"lambda[2]"]
n.samples <- length(lambda1.samples); n.samples
which.ones <- seq(1,n.samples, length=500)
some.lam1.samples <- lambda1.samples[which.ones]
some.lam2.samples <- lambda2.samples[which.ones]

plot( which.ones, some.lam1.samples, type='l',
      xlab="iteration", ylab="lambda[1]")
plot( which.ones, some.lam2.samples, type='l',
      xlab="iteration", ylab="lambda[2]")
rejectionRate(my.coda.samples)

hist(lambda1.samples, prob=TRUE, main="")
lines(density(lambda1.samples))
hist(lambda2.samples, prob=TRUE, main="")
lines(density(lambda2.samples))
summary(my.coda.samples)
effectiveSize(my.coda.samples)
\end{verbatim}

\item Report your results: traceplots, density plots,
summary statistics (including MC error!), and credible intervals. 
What is the effective sample size for each parameter? What is
the rejection rate? Can you tell whether the steps are Gibbs steps (or
something other than Gibbs steps, such as perhaps Metropolis steps)?

%{\bf{When you plot the data, make the plot short but wide, for example,
%by including a statement such as}}
% \verb+par(mai=c(2.5,1,2.5,1))+ 
%{\bf{right before the plot statement.}}

\pagebreak

\item Next you'll fit a model in which $k$, the changepoint year, is a parameter 
to be estimated:

$$Y_i \sim \left\{
\begin{array}{ll}
{\mbox{Poisson}}(\lambda_1), & i = 1,2,\ldots,k \\
{\mbox{Poisson}}(\lambda_2), & i = k+1,k+2,\ldots,n
\end{array}\right.$$
$$\lambda_1 \sim {\mbox{Gamma}}(0.5,3.5), \lambda_2 \sim {\mbox{Gamma}}(0.5,1.6)$$
$$k \sim {\mbox{Discrete uniform}} \{1,2,\ldots,n \}$$

\vspace{.05in}
Note: This problem is a more-than-slightly modified version of Problems 3.9 and 3.10 of
Carlin \& Louis, 3rd ed.; I've chosen simplified priors for $\lambda_1$ and $\lambda_2$,
which is why the values may seem a bit odd.

Please state the posterior distribution,
$$p(\lambda_1,\lambda_2,k) 
\propto L(\lambda_1,\lambda_2,k)\pi(\lambda_1,\lambda_2,k) \,=\, ?$$
Note that $k$ will simply appear in the posterior distribution as part of an
``indicator function'', e.g. $\frac{1}{n}I( 1 \le k \le n)$, to make sure
that $k$ stays between 1 and $n$.

\item Here's the model statement:
\begin{verbatim}
model{
  for( i in 1:n ) {
    disasters[i] ~ dpois( lambda[ idx[i] ] )
    idx[i] <- 1 + step( i-k-0.5 )
    punif[i] <- 1/n
  }
  k ~ dcat( punif[] )
  lambda[1] ~ dgamma(0.5,3.5)
  lambda[2] ~ dgamma(0.5,1.6)
}
\end{verbatim}


Fit this model using BUGS and report your results -- traceplots, density plots,
summary statistics (including MC error!), and credible intervals. 

Note that $k$ is now a parameter rather than data, so you need to remove it
from the list of data and add it to the list of values to initialize.

Be sure to include plots for $k$ as well as $\lambda_1$ and $\lambda_2$. 

\item Comment  on what this second model tells us. Was our initial guess of $k=40$ justified?
Does the data unequivocally support a changepoint at $k=40$?

\item Based on DIC, which statistical model is preferred for these data? The model with
$k=40$ or the model where $k$ is sampled as a parameter. (I'm a bit puzzled as to 
how DIC is calculated for the second model, since the posterior mean of $k$ is not
an integer; perhaps JAGS uses the (rounded value of the) posterior median for its DIC calculations.)

\item Refit your model after including an additional calculation in your model
statement:  \verb+R <- lambda[2] / lambda[1]+

What does \verb+R+ represent? Please provide summary statistics for it, and
discuss briefly.

\end{enumerate}


\item How does the use of \verb+dcat+ in problems 1(c), 1(d), and 2(h) differ
from the use of \verb+dcat+ on page 146 of the lecture notes?
\end{enumerate}





\end{document}
